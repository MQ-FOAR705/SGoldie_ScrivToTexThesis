\documentclass{article}
\usepackage{geometry}
\usepackage[utf8]{inputenc}
\usepackage{graphicx}
% package and file page setting for image input
\graphicspath{ {images/} }

%Allows for linking within document and externally
\usepackage{hyperref}
%setup for hyperref colour schemes
\hypersetup{
    colorlinks=true,
    linkcolor=blue,
    filecolor=magenta,      
    urlcolor=cyan,
    pdftitle={Sharelatex Example},
    pdfpagemode=FullScreen,
    }

\usepackage{geometry}
% package for setting page format and size, and setting paragraph indents, paragraph spacing, and line spacing. 
\geometry{a4paper, left=25mm, right=25mm, top=25mm, bottom=25mm}
\setlength{\parindent}{2em}
\setlength{\parskip}{1em}
\renewcommand{\baselinestretch}{1.3}
% for Baseline stretch
% 1 = 1 line spacing in Word 
% 1.3 = 1.5 line spacing in Word 
% 1.5 = double spacing in Word

\usepackage{fancyhdr}
%package to include header and footer information. Only starts from second page with these settings.
\pagestyle{fancy}
\fancyhf{}
\rhead{Proof of Concept - Planning to Publish}
\lhead{S. Goldie - 42611814}
\lfoot{FOAR705 - Digital Humanities}
\rfoot{Page \thepage}




\title{Proof of Concept - Planning to Publish}
\author{Sheriden Goldie}
\date{Session 2, 2019}

\begin{document}

\maketitle

\tableofcontents
%creates a table of contents based on sections and subsections

\pagebreak
%ends text on current page, and moves all following text to the next page.

\section{Introduction}

This project walk through is designed to give you the basic understanding of how to use Scrivener to create a .tex compatible file. This file can be manipulated to create a document with professional typesetting, consistently and efficiently. 

\section{Requirements}

Before beginning the project walk through, please ensure you have the following programs installed:

\begin{itemize}
    \item Scrivener - This can be downloaded from \url{https://www.literatureandlatte.com/}
    \item TexLive - This can be downloaded from \url{https://www.tug.org/texlive/}
\end{itemize}

Scrivener is a proprietary software, but you can try it for free for 30 non-consecutive days, and this trial version is a complete software download. 

TexLive is a free, open-source software that has significant technical support available online through the website listed above, and services like \url{https://tex.stackexchange.com/}.

\section{Walkthrough}

\subsection{Scrivener}



Create organisation set up in Scrivener
Save as template file

Test import of template file

Export to Tex markup
Test import of Tex file into Tex live and overleaf

Customise memoir class to create thesis appropriate layout


A user will be able to:
Download and install Scrivener and texlive
Import the template organisational structure into Scrivener
Enter information into Scrivener
Load Compile settings to Scrivener
Compile document from Scrivener to Tex


Open Tex file in texLive
Compile the document in texLive 

View a well presented PDF of their document.

\end{document}
